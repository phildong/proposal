\documentclass[master.tex]{subfiles}

\begin{document}

\section*{Specific Aims}

Understanding how memories experienced across large time scales are associated
together is crucial to understanding episodic memories, since the ability to
relate episodes across days is essential to the formation of memory. Recently it
has been demonstrated in rodents that two neutral contexts experienced close in
time shared a larger proportion of neural ensemble than those experienced more
distant in time. Furthermore, subsequent fear conditioning in the later context
increased animals' freezing level in the previous context, suggesting a transfer
of fear memory retrospectively. Such results suggest a linking of two temporally
distinct memories through overlapping neuronal ensembles, and the phenomenon is
termed memory linking. One of the hypothesis that could explain memory linking
is the excitability hypothesis, which states that the neurons encoding an
earlier memory have a transient increase in excitability, making them more
likely to be active during the encoding of a later memory, thus resulting in an
increase of ensembles overlap between the two memories. Such increase in the
ensembles overlap, in turn, may drive memory linking.

However, the ethological implication of memory linking remains unclear. It can
be speculated that memory linking may play a role in causal inference, in that
linking a traumatic experience with an earlier memory may help the animal learn
the causal relationship between memories and avoid future traumatic events.
Thus, retrospective memory linking, where the animal link a traumatic experience
with a previous neutral contextual memory, should be more important than
prospective memory linking, where the animal link a traumatic experience with a
neutral memory that happens later in time. Specifically, the temporal window of
memory linking, which is the maximum time interval within which two memories
could be linked together, should be longer for retrospective memory linking than
for prospective memory linking, since it helps the animal to gather more
information about the potential cause of the traumatic event. However, the
excitability hypothesis would predict the opposite --- Since it has been shown
that negative valence increase neuron excitability, it is expected that neurons
engaged in a negative memory sustain an elevated level of excitability longer
than those in a neutral memory. Thus, for a negative memory, the excitability
model would predict a longer temporal window for prospective memory linking
comparing to retrospective memory linking. We have preliminary results
suggesting a longer temporal window for retrospective memory linking. Thus, the
first goal of the presented proposal is to study the temporal window regarding
the direction of memory linking, and we hypothesize that retrospective memory
linking has a longer temporal window than prospective memory linking.

Moreover, it is unclear how and when memory linking happens. Regarding the first
question, previously, it is believed that memory linking is mediated by overlaps
between two neural ensembles, \textit{i.e} the shared neurons that are active in
both memory. However, it remains under--studied what information do these
overlapping neurons encode, and why they could serve as nodes for memory
linking. The ``memory space'' model suggests that the overlapping cells encode
stimulus or features that are common across episodes. To test this idea, we will
employ dimension reduction algorithms such as PCA to uncover underlying
principal components within episodes. Then we can investigate which components
are shared across linked memories and what is the behavioral correlate, which
would in turn give us a sense of what is the information encoded by the
overlapping neurons. Regarding the second question, the excitability hypothesis
predict that an increase in ensemble overlap for linked memories should be
observed during encoding phase of the memories. However, we have preliminary
data suggest that the difference of the ensemble overlaps between the linked
versus non linked memories are only observed during the retrieval test, but not
during initial encoding. This result suggest that the neural ensemble changed
between encoding and retrieval to drive memory linking, which is likely related
to fear conditioning between encoding and retrieval. Thus we hypothesize that
fear conditioning in one context result in a change in neural ensemble of that
context, which in turn result in an increase in overlap between the linked
contexts and drive memory linking. To test this hypothesis, we will carry out
imaging studies and compare the overlap between ensembles of the same context
before and after fear conditioning. Taken together, the second goal of presented
proposal is to study the neuronal correlate of memory linking regarding the
content of ensemble overlaps as well as the phase at which they are altered.

Lastly, calcium imaging in behaving animals using miniature microscope is an
essential tool for proposed studies. However, one of the challenge facing this
approach is the analysis of imaging data. The constrained non--negative matrix
factorization (CNMF) algorithm performed well on extracting calcium traces from
raw video. However, a lack of user--friendly interface, batch--processing
capability as well as visualization tools of the results limits its popularity
among the community. Thus, the third goal of the presented proposal is to
develop an analysis pipeline that provide an interface to the CNMF algorithm
that enable the user to easily batch-process raw imaging data and visualize the
results to quickly assess the quality of extraction. Such a tool will be very
useful to the scientific community that employ calcium imaging with miniature
microscope during their research.

\paragraph{Aim 1: Study the temporal window of prospective and retrospective
  memory linking.} Test the hypothesis that temporal window for retrospective
memory linking is longer comparing to prospective memory linking.

\paragraph{Aim 2: Study the neuronal correlate of memory linking.} Study the
information encoded by the overlapping neurons in the ensembles of linked
memories. Study the time course of the change in neural ensembles during memory
linking.

\paragraph{Aim 3: Develop analysis pipeline for calcium imaging with miniature microscope.}

\newpage

\end{document}
