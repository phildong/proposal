\documentclass[master.tex]{subfiles}

\begin{document}

\section*{Specific Aims}

Understanding how memories experienced across large time scale are associated
together is crucial to understanding episodic memories. Recently it has been
demonstrated in rodents that two neutral contexts experienced closer in time
shared a larger proportion of neural ensemble. Furthermore, subsequent fear
conditioning in the second context increased animals' freezing level in the
first context, suggesting a transfer of fear memory retrospectively
\cite{cai_shared_2016}. These results showed that memories that have a small
temporal distance can be linked together. However, it remains unclear what
factors may affect the temporal window of memory linking. Specifically, it is
unknown whether the affective value of a memory influence the time window within
which it may be linked to a previous memory. Moreover, it is not clear whether
memory linking can happen prospectively, where the associated fear of previous
context may be transferred to a later context. Furthermore, if there is
prospective memory linking, it is interesting to see whether the temporal window
of prospective memory linking is similar to those in retrospective linking,
\textit{i.e} whether memory linking is symmetric regarding the temporal order of
memories. Thus, the main goal of this proposal is to study how affective value
and temporal order affect the time window of memory linking.

In addition, the analysis of neural dynamic in memory linking experiments have
been limited to comparing the number of overlapping active ensemble cells across
different sessions. Although such analysis successfully provided strong support for
behavior results, it reduces time dimension to a binary, all-or-none
representation, thus precluding the possibility of understanding the temporal
structure of ensemble as well as the evolving nature of population coding within
session. By applying dimension reduction analysis such as principal component
analysis (PCA), we can uncover the underlying structures of neural ensembles for
individual sessions and compare their similarity across linking memories versus
non-linking memories.

\subparagraph{Aim 1: Test the hypothesis that negative valued memories have
  extended retrospective linking window comparing to neutral memories.} It has
been shown that two neutral contexts can be linked together when they are
separated 5 hours apart, but not when they are 2 days apart. We have preliminary
results suggesting that negative-valued context can be linked with a neutral
context two days ago, and they have larger proportion of overlapping ensemble
cells comparing to two neutral contexts. To test this hypothesis, we will use
contextual fear conditioning along with \textit{in vivo} calcium imaging in
freely moving animals.

\subparagraph{Aim 2: Test the hypothesis that prospective memory linking has a
  different temporal window comparing to retrospective memory linking.} It is
hypothesized that memory linking might be mediated through a sustained increase
in neuronal excitability after the first memory. If such hypothesis is true, we
should expect similar temporal window of memory linking regardless of which of
the two memories is later associated with fear, as long as the temporal distance
of the two memories stay the same. However, we have preliminary results showing
that associating the first context with foot-shock did not induce increase of
freezing level in the second context at 5 hours time interval, suggesting that
prospective memory linking is not observed at time scales that induced
retrospective memory linking.

\subparagraph{Aim 3: Test the hypothesis that linked memories have higher
  similarity of ensemble structures.} To test this hypothesis, we can apply PCA
analysis to calcium traces recorded at different behavior sessions. The
resulting principal components can be thought of as subset of cells that exhibit
highly correlated firing. We can then calculate a correlation of the components
across different sessions, and compare the correlation between linking contexts
with those between non-linking contexts. We predict that the correlation of
structured ensemble components are higher for linking contexts comparing to
non-linking contexts.

\end{document}