\documentclass[master.tex]{subfiles}

\begin{document}

\section*{Specific Aims}

Understanding how memories experienced across large time scales are associated
together is crucial to understanding episodic memories, since the ability to
relate episodes across days is essential to the formation of memory. Recently it
has been demonstrated in rodents that two neutral contexts experienced close in
time shared a larger proportion of neural ensemble than those experienced more
distant in time . Furthermore, subsequent fear conditioning in the second
context increased animals' freezing level in the first context, suggesting a
transfer of fear memory retrospectively \cite{cai_shared_2016}. These results
suggest that memories that have a small temporal distance can be linked
together. However, it remains unclear what factors may affect the temporal
window of memory linking. Specifically, it is unknown whether the affective
value of a memory influences the time window within which it may be linked to a
previous memory. Moreover, it is not clear whether memory linking can happen
prospectively, where the associated fear of a previous context may be
transferred to a later context. Furthermore, if there is prospective memory
linking, it is interesting to see whether the temporal window of prospective
memory linking is similar to those in retrospective linking, \textit{i.e}
whether memory linking is symmetric regarding the temporal order of memories.
Thus, the main goal of this proposal is to study how affective value and
temporal order affect the time window of memory linking, with both behavioral
experiments and calcium imaging in behaving animals.

It has been shown that two neutral contexts can be linked together when they are
separated by 5 hours, but not when they are separated by 2 days. We have
preliminary results suggesting that negative-valued context can be linked with a
neutral context 2 days ago, and they have larger proportion of overlapping
ensemble cells comparing to two neutral contexts. Thus, our first hypothesis is
that negative emotional value extend the temporal window of memory linking
retrospectively. On the other hand, we have preliminary results showing that a
negative-valued context can link both backward and forward with a neutral
context across 5 hours time interval, but can only link backward, but not
forward, across 1 day and 2 days interval. Thus, our second hypothesis is that
negative-valued memory have a longer retrospective memory linking window
compared to prospective linking window. To test these hypothesis, we will carry
out behavior experiments utilizing contextual fear conditioning, as well as
\textit{in vivo} calcium imaging in freely moving animals.

In addition, the analysis of neural dynamic in memory linking experiments have
been limited to comparing the number of overlapping active ensemble cells across
different sessions. Although such analysis provided a strong correlate of
behavior results, it reduces time dimension to a binary, all-or-none
representation, thus precluding the possibility of understanding the temporal
structure of ensemble as well as the evolving nature of population coding within
session. By applying dimension reduction analysis such as principal component
analysis (PCA), we can uncover the underlying structures of neural ensembles for
individual sessions and compare their similarities across linking memories
versus non-linking memories. Our hypothesis is that the linked memories have
higher similarities in ensemble structures compared to non-linked memories. To
test this hypothesis, we can apply PCA analysis to calcium traces recorded at
different behavior sessions. The resulting principal components can be thought
of as subset of cells that exhibit highly correlated firing. We can then
calculate a correlation of the components across different sessions, and compare
the correlation between linking contexts with those between non-linking
contexts. We predict that the correlation of structured ensemble components are
higher for linking contexts comparing to non-linking contexts.

\paragraph{Aim 1: Study the behavioral effect of emotional value and temporal
  order on memory linking.}

\subparagraph{Aim 1a:} Test the hypothesis that negative emotional value extend
retrospective memory linking window.

\subparagraph{Aim 1b:} Test the hypothesis that negative valued memory has a
longer retrospective memory linking window compared to prospective memory
linking.

\paragraph{Aim 2: Test the hypothesis that overlap in neural ensembles correlate
  with the behavior effect of emotional value and temporal order on memory
  linking.}

\paragraph{Aim 3: Test the hypothesis that linked memories have higher
  similarity of ensemble structures compared to non-linked memories.}

\newpage

\end{document}
