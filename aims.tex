\documentclass[master.tex]{subfiles}

\begin{document}

\section*{Specific Aims}

Understanding how memories experienced across large time scales are associated
together is crucial to understanding episodic memories, since the ability to
relate episodes across days is essential to the formation of memory. Recently it
has been demonstrated in rodents that two neutral contexts experienced close in
time shared a larger proportion of neural ensemble than those experienced more
distant in time. Furthermore, subsequent fear conditioning in the later context
increased animals' freezing level in the previous context, suggesting a transfer
of fear memory retrospectively. Such results suggest a linking of two temporally
distinct memories through overlapping neuronal ensembles, and the phenomenon is
termed memory linking. One of the hypothesis that could explain memory linking
is the excitability hypothesis, which states that the neurons encoding an
earlier memory have a transient increase in excitability, making them more
likely to be active during the encoding of a later memory, thus resulting in an
increase of ensembles overlap between the two memories. Such increase in the
ensembles overlap, in turn, may drive memory linking.

However, the ethological implication of memory linking remains unclear. It can
be speculated that memory linking may play a role in causal inference, in that
linking a traumatic experience with an earlier memory may help the animal learn
the causal relationship between memories and avoid future traumatic events.
Thus, retrospective memory linking, where the animal link a traumatic experience
with a previous neutral contextual memory, should be more important than
prospective memory linking, where the animal link a traumatic experience with a
neutral memory that happens later in time. Specifically, the temporal window of
memory linking, which is the maximum time interval within which two memories
could be linked together, should be longer for retrospective memory linking than
for prospective memory linking, since it helps the animal to gather more
information about the potential cause of the traumatic event. However, the
excitability hypothesis would predict the opposite --- Since it has been shown
that negative valence increase neuron excitability, it is expected that neurons
engaged in a negative memory sustain an elevated level of excitability longer
than those in a neutral memory. Thus, for a negative memory, the excitability
model would predict a longer temporal window for prospective memory linking
comparing to retrospective memory linking. We have preliminary results
suggesting a longer temporal window for retrospective memory linking. Thus, the
first goal of the presented proposal is to study the temporal window regarding
the direction of memory linking, and we hypothesize that retrospective memory
linking has a longer temporal window than prospective memory linking.

The second goal of our proposal is to study how ensemble dynamic of two
contextual memories contribute to the emergence of the increase in overlap
between the two ensembles, which in turn drives memory linking. We have
preliminary results suggesting that the increase in ensemble overlap between the
two contexts emerges only during retrieval testing, but not during encoding.
Thus, an intuitive hypothesis is that the representation of one of the contexts
changed after encoding, so that it is more similar to the other context. To test
this hypothesis, we will investigate the reactivation rate for each context,
which is the amount of overlap between the ensembles of encoding and retrieval
of the same context. We will also investigate the overlap between the ensembles
of one context during retrieval and the other context during encoding, which
indicate whether one of the context become more similar to the other after
encoding. In addition, we will investigate whether different ensembles during
encoding have an equal contribution to the overlap between the two ensembles
during retrieval. To test this, for each ensemble during encoding, we will
calculate a rate of neurons that is also active in both contexts during
retrieval. We can then compare the rates between different ensembles during
encoding to see whether they contribute equally to the overlap between the two
ensembles in retrieval. Lastly, the excitability hypothesis predict a
correlation between the activity level of a neuron and the likelihood of the
neuron being active again in future sessions. To test this, for any two
recording sessions, we will classify the neurons in the earlier session into two
populations according to whether they are also active during the later session.
We can then compare the mean normalized activity level of the two population to
see whether the activity level of a neuron could predict whether they are active
in future sessions. Taken together, these tests will give us more insights in
how the neurons in different ensembles and their temporal dynamic contribute to
the emergence of memory linking.

Lastly, calcium imaging in behaving animals using miniature microscope is an
essential tool for proposed studies. However, one of the challenge facing this
approach is the analysis of imaging data. The constrained non--negative matrix
factorization (CNMF) algorithm performed well on extracting calcium traces from
raw video. However, a lack of user--friendly interface, batch--processing
capability as well as visualization tools of the results limits its popularity
among the community. Thus, the third goal of the presented proposal is to
develop an analysis pipeline that provide an interface to the CNMF algorithm
that enable the user to easily batch-process raw imaging data and visualize the
results to quickly assess the quality of extraction. Such a tool will be very
useful to the scientific community that employ calcium imaging with miniature
microscope during their research.

\paragraph{Aim 1: Study the temporal window of prospective and retrospective
  memory linking.} Test the hypothesis that temporal window for retrospective
memory linking is longer comparing to prospective memory linking.

\paragraph{Aim 2: Study how ensemble dynamic contribute to memory linking.} Test
whether ensemble of one context become more similar to the other linked context
after encoding. Test whether different ensembles during encoding have equal
contribution to the overlap between ensembles of linked context during
retrieval. Test whether activity level of neurons in one session could predict
whether they are active in future sessions.

\paragraph{Aim 3: Develop analysis pipeline for calcium imaging with miniature
  microscope.}

\newpage

\end{document}
