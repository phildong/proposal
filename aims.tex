\documentclass[master.tex]{subfiles}

\begin{document}

\section*{Specific Aims}

Understanding how memories experienced across large time scales are associated
together is crucial to understanding episodic memories, since the ability to
relate episodes across days is essential to the formation of memory. Recently it
has been demonstrated in rodents that two neutral contexts experienced close in
time shared a larger proportion of neural ensemble than those experienced more
distant in time. Furthermore, subsequent fear conditioning in the later context
increased animals' freezing level in the previous context, suggesting a transfer
of fear memory retrospectively. Such results suggest a
linking of two temporally distinct memories through overlapping neuronal
ensembles, and the phenomenon is termed memory linking. However, it remains
unclear what factors may affect the temporal window of memory linking, which is
the maximum time interval within which two memories can be linked together.
Specifically, it is unknown whether the affective valence of a memory influences
the extent or the symmetry of the temporal window of memory linking. For
instance, it is unclear whether negative valence of a memory could extend the
time window within which it may be linked to a previous memory. Moreover, it is
not clear whether a negative memory could link forward with a memory that
happens later in time, (\textit{i.e} linking to a memory prospectively).
Furthermore, it's unclear whether such prospective memory linking has a similar
time window as retrospective memory linking (\textit{i.e} whether the time
window is symmetric regarding the temporal order of memories). Thus, the main
goal of this proposal is to study how affective value and temporal order affect
the temporal window of memory linking, with both behavioral experiments and
calcium imaging in behaving animals.

One of the hypothesis that could explain memory linking is the excitability
hypothesis, which states that the neurons encoding an earlier memory have a
transient increase in excitability, making them more likely to be active during
the encoding of a later memory, thus resulting in an increase of ensembles
overlap between the two memories. Such increase in the ensembles overlap, in
turn, may drive memory linking. Since it has been shown that negative valence
increase neuron excitability, it is expected that neurons engaged in a negative
memory sustain an elevated level of excitability longer than those in a neutral
memory. Thus, for a negative memory, the excitability model would predict a
longer temporal window for prospective memory linking comparing to retrospective
memory linking. Furthermore, the excitability model would not predict a change
of retrospective memory linking window as a function of emotional valence.
However, from an ethological point of view, retrospective memory linking is more
important than prospective memory linking, since past memories may have a causal
contribution to future events, but not the other way around. Moreover, memory
linking of more negative events should extend further back in time, since that
helps animals to gather more information to avoid future traumatic events. In
fact, we have preliminary data suggesting that negative valence increase the
retrospective memory linking window, and that retrospective memory linking
window is longer than prospective linking window. Thus, we hypothesis that
negative emotional valence extend retrospective memory linking time window, and
that retrospective memory linking time window is longer than those in
prospective memory linking. To test these hypothesis, we will carry out behavior
experiments utilizing contextual fear conditioning, as well as \textit{in vivo}
calcium imaging in freely moving animals.

Besides behavior, the other important aspect of memory linking is an increase of
overlaps between the ensembles of two linked memories. Thus, our second goal is
to study the neural correlates of memory linking during the manipulation of
emotional valence and temporal order of contexts. To achieve this goal, we will
carry out calcium imaging in behaving animals. One of the difficulties facing
this approach is the analysis of imaging data. Various algorithms have been
developed to extract calcium traces from raw videos, however a user-friendly
pipeline is lacking. Thus our first step towards this goal is to develop
analysis pipeline that can visualize the extraction results and register neurons
across recording sessions, so that we may compare neural ensembles across
sessions. We hypothesize that if the temporal window of memory linking changed,
and transfer of fear is observed between two contexts, the overlaps between the
ensembles encoding the two contexts should also increase. In addition, by
applying dimension reduction analysis such as principal component analysis
(PCA), we can uncover the underlying temporal structures of neural ensembles for
individual sessions. Thus we also hypothesize that the similarities between
principal components of two ensembles are higher if the corresponding contexts
are linked together. In other words, the similarities between principal
components of two memory ensembles should be consistent with the overlaps
between the two ensembles.

\paragraph{Aim 1: Study the effect of emotional value on the extent and symmetry
  of the temporal window of memory linking.} Test the hypothesis that negative
emotional value extend retrospective memory linking time window, and that
retrospective memory linking time window is longer than those in prospective
memory linking.

\paragraph{Aim 2: Study the neuronal correlate of memory linking.} Develop
analysis pipeline for calcium imaging data. Test the
hypothesis that linked memories have larger neural ensemble overlaps, and that
linked memories also has more similarities in ensemble structures.

\newpage

\end{document}
