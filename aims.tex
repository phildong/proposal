\documentclass[master.tex]{subfiles}

\begin{document}

\section*{Specific Aims}

Understanding how memories experienced across large time scales are associated
together is crucial to understanding episodic memories, since the ability to
relate episodes across days is essential to the formation of memory. Recently it
has been demonstrated in rodents that two neutral contexts experienced close in
time shared a larger proportion of neural ensemble than those experienced more
distant in time. Furthermore, subsequent fear conditioning in the later context
increased animals' freezing level in the previous context, suggesting a transfer
of fear memory retrospectively \cite{cai_shared_2016}. Such phenomenon is coined
as temporal memory linking. However, it remains unclear what factors may affect
the temporal window within which two memories can be linked together.
Specifically, it is unknown whether the affective valence of a memory influences
the extent or the symmetry of the temporal window of memory linking. For
instance, it is unclear whether negative valence of a memory could extend the
time window within which it may be linked to a previous memory. Moreover, it is
not clear whether a negative memory could link forward with a memory that
happens later in time, (\textit{i.e} linking to a memory prospectively).
Furthermore, it's unclear whether such prospective memory linking has a similar
time window as retrospective memory linking (\textit{i.e} whether the time
window is symmetric regarding the temporal order of memories). Thus, the main
goal of this proposal is to study how affective value and temporal order affect
the temporal window of memory linking, with both behavioral experiments and
calcium imaging in behaving animals.

Two models have been developed to explain the memory linking phenomenon from two
perspective: the excitability hypothesis from cellular level and temporal
context model from computational level. For a negative memory, both model would
predict a stronger prospective memory linking than retrospective memory linking.
At the same time, neither model would readily expect a change of retrospective
memory linking window as a function of emotional valence. However, from an
ethological point of view, retrospective memory linking is more important than
prospective memory linking, since past memories may have a causal contribution
to future events, but not the other way around. Moreover, memory linking of more
negative events should extend further back in time, since that helps animals to
gather more information to avoid future traumatic events. Thus, we hypothesis
that negative emotional valence extend retrospective memory linking time window,
and that retrospective memory linking time window is longer than those in
prospective memory linking. To test these hypothesis, we will carry out behavior
experiments utilizing contextual fear conditioning, as well as \textit{in vivo}
calcium imaging in freely moving animals.

In addition, the analysis of neural dynamic in memory linking experiments have
been limited to comparing the number of overlapping active ensemble cells across
different sessions. Although such analysis provided a strong correlate of
behavior results, it reduces time dimension to a binary, all-or-none
representation, thus precluding the possibility of understanding the temporal
structure of ensemble as well as the evolving nature of population coding within
session. By applying dimension reduction analysis such as principal component
analysis (PCA), we can uncover the underlying structures of neural ensembles for
individual sessions and compare their similarities across linking memories
versus non-linking memories. Our hypothesis is that the linked memories have
higher similarities in ensemble structures compared to non-linked memories. To
test this hypothesis, we can apply PCA analysis to calcium traces recorded at
different behavior sessions. The resulting principal components can be thought
of as subset of cells that exhibit highly correlated firing. We can then
calculate a correlation of the components across different sessions, and compare
the correlation between linking contexts with those between non-linking
contexts. We predict that the correlation of structured ensemble components are
higher for linking contexts comparing to non-linking contexts.

\paragraph{Aim 1: Study the effect of emotional value on the extent and symmetry
  of the temporal window of memory linking.} Test the hypothesis that negative
emotional value extend retrospective memory linking time window, and that
retrospective memory linking time window is longer than those in prospective
memory linking.

\paragraph{Aim 2: Study the neuronal correlate of memory linking.} Test the
hypothesis that linked memories have larger neural ensemble overlaps, and that
linked memories also has more similarities in ensemble structures.

\newpage

\end{document}
