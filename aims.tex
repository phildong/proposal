\documentclass[master.tex]{subfiles}

\begin{document}

\section*{Specific Aims}

Understanding how experiences separated by time are associated together in
memory is crucial since memories for individual events can only be understood in
the framework of one's collective experience. Recently it was demonstrated in
rodents that two neutral contexts experienced close in time can be bound in
memory, a phenomenon termed memory linking. The goal of this proposal is to
understand the ethological function of memory linking and how memory linking is
supported by hippocampal ensembles.

It can be speculated that memory linking provides animals with the ability to
predict environmentally relevant events, such as danger. For example, linking an
aversive experience with an earlier experience may help the animal avoid future
aversive events. If true, the temporal window of retrospective memory linking,
where an animal links an aversive experience with a previously experienced
neutral stimulus, should be greater than those for prospective memory linking,
where an animal links an aversive experience with a neutral event experienced
later in time. Notably, this ethological prediction differs dramatically from
the prediction of the excitability hypothesis for memory linking, a previously
proposed neurobiological explanation of memory linking. The excitability
hypothesis states that the neurons encoding a memory have a transient increase
in excitability, making them more likely to be incorporated into the ensemble of
a memory formed shortly thereafter. Since negative valence increases neuron
excitability, neurons engaged in a negative memory presumably sustain an
elevated level of excitability for longer than those in a neutral memory. Thus,
for a negative memory, the excitability hypothesis predicts a longer temporal
window for prospective memory linking. \textbf{The first aim of this proposal is
  to study the temporal window regarding the direction of memory linking.} In
line with recently acquired preliminary data, we hypothesize that retrospective
memory linking has a longer temporal window.

\textbf{The second aim of this proposal is to study the neural ensembles during
  memory linking.} In order to do this, we will carry out calcium imaging in
behaving animals using miniature microscopes. One of the challenges facing this
approach is the analysis of imaging data. A non--negative matrix factorization
(CNMF) algorithm has been shown to perform well on extracting calcium traces
from raw video. However, lack of a user--friendly interface, batch--processing
capability, and visualization tools of the results limits the feasibility and
efficiency of CNMF in practice. \textbf{Thus, as the first part of the second
  aim, we will develop an analysis pipeline} that provides an interface to the
CNMF algorithm enabling the user to easily batch--process raw imaging data and
visualize the results to quickly assess the quality of extraction.

\textbf{The second part of the second aim is to study how the individual
  ensemble dynamics of two contextual memories contribute to the emergence of
  shared neuronal ensemble that supports memory linking.} We have preliminary
results suggesting that an increase in ensemble overlap for the linked contexts
emerges during testing, but not during initial encoding. Thus, an intuitive
hypothesis is that the representation of one of the contexts changes after
encoding, so that it is more similar to the other context. We will investigate
the reactivation rate for each context, which is the amount of overlap between
the ensembles during retrieval and encoding. We hypothesize that one of the two
ensembles will become more similar to the other. Moreover, we will study whether
different ensembles during encoding contribute differently to the overlap
between ensembles during retrieval. Furthermore, we will assess whether
activation patterns during initial encoding predict which cells are recruited
into the shared neural ensemble for two linked contexts.

\paragraph{Aim 1: Study the temporal window of prospective and retrospective
  memory linking.}

Test the hypothesis that temporal window for retrospective memory linking is
longer compared to prospective memory linking.

\paragraph{Aim 2: Study neural ensemble during memory linking.}

\paragraph{Aim 2.1: Develop analysis pipeline for calcium imaging with miniature
  microscope.}

\paragraph{Aim 2.2: Study how ensemble dynamic contribute to memory linking.}

Test whether ensemble of one context become more similar to the other linked
context after encoding. Test whether different ensembles during encoding have
equal contribution to the overlap between ensembles of linked context during
retrieval. Test whether activity level of neurons in one session could predict
whether they are active in future sessions.

\newpage

\end{document}
