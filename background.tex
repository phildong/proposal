\documentclass[master.tex]{subfiles}

\begin{document}

\section*{A. Significance}

It is generally believed that hippocampus make an important contribution to
episodic memories in rodents. Traditionally, studies on hippocampal neuronal
coding have been focused on how information are encoded in the \textbf{activity
  pattern} of neurons. Usually, such studies involve measuring neuronal
activities during repeated retrieval of same memory, and measure how the
activities of neurons respond consistently to a behavior variable across
retrieval sessions. For instance, ``place cells'' have been found to encode the
location of the animal in a familiar environment \cite{hartley_space_2014}, and
``time cells'' have been found to encode elapsed time during well--learned,
time--dependent tasks \cite{eichenbaum_time_2014}. On the other hand, another
trend of studies have focused on how information might be encoded in the
\textbf{identity} of neurons. For example, the idea of ``neural ensemble''
states that different population of neurons are engaged in the encoding of
different memories, and memory retrieval happens through the reactivation of the
corresponding neural population. The population of neurons that encode a memory
is thus termed the ensemble of the said memory. Indeed, it is found that
artificially stimulating an ensemble is sufficient to drive the retrieval of the
corresponding memory and elicit behavior response \cite{ramirez_creating_2013}.
The two distinct but not mutually exclusive types of hippocampal coding ---
through activity pattern and through neuron identity --- can be brought together
by a conceptual framework coined as ``memory space''
\cite{eichenbaum_hippocampus_1999}. Briefly, it is hypothesized in the ``memory
space'' concept that the activities of cells encode details of a memory episode,
such as stimulus, location and time, whereas the common cells that are shared
across episodes, or the overlapping neuron population between ensembles, may
serve as ``nodes'' that bridge together different memory episodes. In other
words, the neuronal activities encode information within episodes, while the
identity of ensembles encode relationship between episodes. Consistent with this
concept, it is found that the ensembles of the \textbf{same} familiar
environment ``drift'' across time, so that different but overlapping ensembles
were activated during the retrieval of memory of the same environment at
different times. More importantly, it is found that the overlap between
ensembles depend on time, in that ensembles of episodes that happened closer in
time share more neurons in common. Moreover, it is found that the subset of
neurons that were active across all retrieval episodes sustain a stable spatial
map of the environment so that the location of the animals can be reliably
decoded with only the activities from this subset of cells. Taken together,
these results suggest that overlapping neurons between ensembles encode
relationship between memory episodes, in that they encode both the information
that is common across episodes (the same spatial environment) and the temporal
relationship of episodes (the temporal distances between them), which is
consistent with the prediction of ``memory space'' concept
\cite{rubin_hippocampal_2015}. However, two important aspects of the ``memory
space'' concept remained untested:
\begin{inparaenum}[a)]
\item whether the ensembles overlap also encode temporal distance between
  memories of \textbf{different} contexts;
\item whether the overlapping ensemble can actually drive the ``bridging'' of
  different memory episodes.
\end{inparaenum}
Understanding how distinct memories can be related together is essential to
understanding episodic memory, since the ability to associate different episodes
across long periods of times is essential to forming memories.

Recently, it has been found in rodent hippocampus that the neuronal ensembles of
two distinct contextual memories separated by 5 hours time interval has more
overlapping cells than those separated by 2 days or 7 days. Interestingly,
subsequent fear conditioning in the later context induce elevated freezing level
in the former context when the two contexts are separated by 5 hours, indicating
a transfer of fear memory from the second context to the first. Such results
suggest a linking of two temporally distinct memories through overlapping
neuronal ensembles, and the phenomenon is termed memory linking
\cite{cai_shared_2016}. Following these findings, two models have been proposed
to explain the phenomenon of memory linking: On cellular and circuit level, It
has been hypothesized that memory linking happens through excitability
mechanism, where the ensemble neurons of first memory sustain an elevated
excitability during the memory linking time window, and thus are more likely to
be recruited during the encoding of the second memory, facilitating the linking
of the two memory \cite{kastellakis_linking_2016}; At the same time, from a
conceptual and computational aspect, temporal context model suggests that memory
linking is a peculiar case of a more general temporal context framework, which
argues that features of memories are associated with an ever-drifting temporal
context, and all recollection of episodic memories, as well as formation of
semantic memories, happen through the retrieval of the associated temporal
context \cite{howard_temporal_2005}.

However, various aspects of memory linking remain under-studied. Most notably,
it is unclear what factors affect the temporal window of memory linking, defined
as the maximum time interval within which two memories could be linked together.
Besides, the effect of temporal order on memory linking remains unclear - it has
been demonstrated that memory linking can happen retrospectively, in that the
fear associated with a later memory can transfer back to a neutral memory that
happened earlier. It is unknown, however, whether memory linking could happen
prospectively, where the fear associated with a memory can transfer forward to a
neutral memory that happens later in time. Moreover, if prospective memory
linking exists, it is interesting to see whether it has a same temporal window
as retrospective memory linking. Taken together, two important and inter-related
questions remain unclear for the memory linking phenomenon:
\begin{inparaenum}[a)] \bfseries
\item whether and how the temporal order of the experiences affect the temporal
  window of memory linking.
\item whether and how negative emotional valence of the experiences affect the
  retrospective temporal window of memory linking.
\end{inparaenum}
Regarding the first question, the excitability hypothesis would predict longer
memory linking window for prospective memory linking, since negative emotional
valence increase the excitability of neurons, making them sustain an elevated
excitability for longer period of time comparing to those engaged in neutral
memories, thus extend the time window where a negative memory could be linked to
a neutral memory in the future, but not in the past. Similarly, the temporal
context model would predict a stronger prospective memory linking as well.
Regarding the second question, the excitability hypothesis would not predict an
effect of negative emotional valence on retrospective memory linking, since the
excitability of neurons engaged in a neutral memory should not be affect by
emotional valence of a memory happens in the future. Meanwhile, the temporal
context model would fail to provide a prediction regarding the second question
since emotional valence has not been integrated into the model.

However, from a ethological perspective, the predictions regarding the two
questions would be different from those predicted by the two existing models.
Specifically:
\begin{inparaenum}[a)]
\item relating a memory to past experiences is more beneficial than relating a
  memory to future experiences, since only past experiences may have a causal
  role which is important to learn.
\item relating a highly traumatic memory to other experiences (especially past
  experiences) is more beneficial than relating a neutral memory to others, so
  that the animal may learn to avoid the same situation in the future.
\end{inparaenum}
An important example of such perspective is conditioned taste aversion, where
the animals learned to associate the negative experience (sickness) with a past
experience (consumption of food), and such association depends on the valence of
the negative valence (how much sickness was induced).

Thus, the main goal of this proposal is to study the effect of temporal order
and emotional valence on memory linking. The study would help us understand the
mechanism and behavioral significance of memory linking, and may extend our
knowledge on episodic memory in general.

In addition to behavior, it is important to see whether there is neural
correlates of memory linking in hippocampus. Traditionally, the analysis of
neural recording data in memory linking experiments have been limited to
comparing overlapping ensemble cells that are active during recording sessions.
Such analysis provided a simple estimation of similarities between ensembles and
successfully supported behavior data. However, it reduces the time dimension of
each recording session to a binary, ``active-or-not'' representation, precluding
any analysis on the structure of the ensemble within session. This limitation is
mainly due to the task-free and one-trial-learning nature of memory linking,
where there is no task variables to align the recording data to, nor is there
enough time for place cells to be formed and detected. Dimension reduction
approaches, or more specifically principal component analysis (PCA) is a very
useful tool in such circumstances, since it can transform higher dimension
recording data to lower dimension, temporally structured components in an
unsupervised manner. Thus another goal of the presented proposal is to apply PCA
analysis to neural recording data during memory linking experiments. Such
analysis could uncover the underlying structures of memory ensembles, and help
us understand the nature of memory linking on the ensemble level.


\end{document}
