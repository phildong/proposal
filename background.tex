\documentclass[master.tex]{subfiles}

\begin{document}

\section*{A. Scientific Premise}

It has been found in mice that two contextual memories for events that happened
close in time can be linked together. This was evidenced by the ensembles of the
two memories sharing a larger proportion of overlapping neurons in the
hippocampus, and fear conditioning in one of the contexts resulting in freezing
behavior in the other \cite{cai_shared_2016}. However, the temporal window for
memory linking, as well as the neuronal basis of this phenomenon remains
under--studied.


\section*{B. Background and Significance}

Understanding how distinct memories can be related together is essential to
understanding episodic memory, since memories for individual events can only be
understood in the framework of one's collective experience. It is generally
believed that different memories are encoded with different populations of
neurons, which are reactivated during memory retrieval
\cite{tayler_reactivation_2013}. Such a specific population of neurons are
termed the ensemble of the memory. Indeed, it has been found that artificially
stimulating an ensemble is sufficient to drive the retrieval of the memory and
elicit the associated behavioral response (e.g. freezing)
\cite{ramirez_creating_2013}. Thus, it can be speculated that two distinct
memories can be associated together by sharing a proportion of neurons that is
common to both ensembles, so that activation of one ensemble might trigger the
activation of the other as well, potentially through a pattern completion
mechanism. Consistent with this idea, it has been found that two types of
aversive memories can be associated together by repeatedly presenting the
stimuli that triggers each of them at the same time
\cite{yokose_overlapping_2017}. Importantly, the overlap between amygdala
ensembles encoding the two aversive memories is larger than those in a control
group that did not undergo the paired presentation of stimuli. Moreover,
artificially suppressing the activity of the amygdala neuron population shared
between the two ensembles disrupted the behavioral association between the two
aversive memories, without affecting the independent recall of each of them
\cite{yokose_overlapping_2017}. Besides explicit pairing of two memories by
triggering the recall of them at the same time, it has also been found that two
fear memories can be associated together if they are encoded close in time.
Specifically, two auditory fear conditioning memories can be associated together
if they are encoded within 6 hours, but not when they are separated by 24 hours.
Consistent with previous findings, the amygdala ensembles of the two memories
that were associated together had a higher proportion of overlap
\cite{rashid_competition_2016}. It has been hypothesized that such a
time-dependent association of different memories is mediated by an
excitability-based memory allocation mechanism: Neurons encoding an earlier
memory sustain an elevated level of excitability for certain period of time,
thus biasing the allocation of a later memory towards the same population of
neurons, resulting in an increase in overlap of the ensembles of the two
memories \cite{lisman_memory_2018, yiu_neurons_2014, zhou_creb_2009}. Indeed, it
has been found that artificially manipulating the amygdala neurons encoding one
memory can bidirectionally bias the allocation of cells to a second memory and
either rescue or disrupt the association between the two memories
\cite{rashid_competition_2016}. Taken together, these studies suggest that
aversive memories can be associated together in a time-dependent manner.
However, it is important to know whether such association can occur naturally
with memories that have neutral emotional valence, which is key to the question
of how episodes of memories that does not have negative emotional valence can be
linked together.

Recently, it was found in the rodent hippocampus that the neuronal ensembles of
two neutral contextual memories separated by 5 hours have greater overlap than
those separated by 2 days or 7 days. Moreover, subsequent fear conditioning in
the latter of the two contexts initially experienced promotes elevated freezing
levels in both that context and the linked context. Such results suggest a
linking of two temporally distinct memories through overlapping neuronal
ensembles. This phenomenon has been termed memory linking
\cite{cai_shared_2016}.

How might neutral contextual memories experienced closer in time come to share
greater overlapping hippocampal ensemble? It has been found that the
representation of a familiar environment in the rodent hippocampus drifts across
time, such that the similarity between ensembles for the same environment at
different times decays as a function of temporal distance
\cite{mankin_neuronal_2012, ziv_long-term_2013-1}. Such drift may reflect
spontaneous turn--over of neural ensembles, possibly driven by fluctuations in
cellular excitability over time. When combined with the excitability hypothesis,
this could serve as a potential mechanism of the memory linking phenomenon. If
two memories happen close in time so that when the second memory happens,
neurons encoding the first memory still have elevated excitability, the two
memories will have larger overlap in ensembles since the allocation of the
second memory is biased towards the same neurons encoding the first memory. If,
however, the two memories are separated by a large time interval, so that the
excitability of neurons encoding the first memory return to baseline before the
second memory happens, the two memories will have chance level overlap in
ensembles. Taken together, these results suggest that neutral memories can be
linked in a time-dependent manner.

However, the temporal window of memory linking, which is the maximum time
interval within which two memories can be linked together, remains
under--studied. This question is important since it is closely related to the
ethological implication of memory linking --- It can be speculated that memory
linking may play a role in learning the relationships between memory episodes,
and the temporal window of memory linking reflects how much information is
gathered to form relational memories for a specific episode. One of the most
important relationship between memory episodes is causal relationship, where
linking an aversive experience with an earlier experience may help the animal
learn the causal relationship between events and avoid future aversive events.
An important example is conditioned taste aversion (CTA), where the animals
learn to associate sickness with the consumption of food several hours earlier,
a period much longer than most stimulus--food associations
\cite{garcia_conditioned_1955}. In addition to highlighting the important
biological function of linking events disparate in time, this example also
highlights the predictive nature such associations can serve --- CTA
specifically ties sickness with stimuli experienced prior to sickness. Thus,
retrospective memory linking, where an animal links an aversive experience with
a memory of a previously experienced neutral context, could be more
ethologically valuable than prospective memory linking, where an animal links an
aversive experience with a neutral contextual memory that happens later in time.
Specifically, the temporal window of memory linking could be longer for
retrospective memory linking than for prospective memory linking, since it might
help the animal to gather more information about the potential cause of the
aversive event.

However, the excitability hypothesis would predict the opposite. Since it has
been shown that memories with negative valence increase neuron excitability
\cite{mckay_intrinsic_2009}, it is expected that neurons engaged in a negative
memory sustain an elevated level of excitability longer than those in a neutral
memory. Thus, for a negative memory, the excitability hypothesis would predict a
longer temporal window for prospective memory linking compared to retrospective
memory linking. In order to study the temporal window of prospective and
retrospective memory linking, we have carried out preliminary behavior studies.
The results suggest a longer temporal window for retrospective memory linking
than for prospective memory linking (\autoref{fig:prelim_pro_retro}). Thus, we
hypothesize that retrospective memory linking has a longer temporal window than
prospective memory linking.

Moreover, it is unclear how memory linking happens. As mentioned before, we have
ethological speculations and preliminary data that cannot be fully addressed by
the excitability hypothesis. We aim to investigate the circuit mechanisms
contribute to memory linking. Specifically, we will focus on studying the neural
correlates and ensemble dynamics of memory linking. Firstly, one of the most
important neural correlates of memory linking is the ensemble overlap between
the linked memories, since it has been shown that manipulating this overlapping
population of neurons directly and specifically affects memory linking
\cite{yokose_overlapping_2017}. Thus, we will measure ensemble overlap between
the neutral and aversive contexts during encoding and retrieval of the contexts.
Our preliminary data suggests that the increase in overlap of hippocampal
ensembles emerges after the encoding of the two memories, which is contrary to
what the excitability hypothesis would predict (\autoref{fig:prelim_val_imag}).
Next, we will examine how the overlapping ensemble between the two memories
emerge, specifically if the representation of the neutral or aversive contexts
shift towards the other, increasing the overlapping ensemble. Lastly, the
excitability hypothesis predicts that the activity level of a neuron during the
encoding of the first memory should determine the likelihood of that neuron to
be allocated to the second memory. However, this correlation has never been
explicitly tested in previous memory linking studies. Thus, we will test whether
the activity level of neurons during the encoding of the first memory is
predictive of whether they will also encode the second memory. Taken together,
these studies will give us more insight into how memory linking happens on a
neural ensemble level.

Lastly, calcium imaging with miniature microscope in behaving animals is an
important tool to study the neuronal dynamics in memory linking studies, due to
its capability to track large populations of neurons across long periods of
time. However, one of the challenges facing this technique is the analysis of
imaging data. Specifically, extracting calcium traces of individual neurons from
raw video is a difficult problem. Several approaches have been developed to
address this difficulty. Most notably, a variant of the constrained
non--negative matrix factorization (CNMF) algorithm has been shown to work well
on calcium imaging data with miniature microscopes
\cite{pnevmatikakis_simultaneous_2016, zhou_efficient_2016}. However, in
practice, the accessibility and efficiency of such approaches has been limited
for several reasons:
\begin{inparaenum}[a)]
\item The effect of different parameters in the mathematical model and the
  complications introduced by performance optimization of the code can be hard
  to comprehend for researchers without a mathematical or programming
  background.
\item Lack of a pre--processing pipeline that specifically addresses various
  artifacts usually observed in miniature microscope data requires the user to
  either rely on other software to conduct pre--processing steps or modify the
  existing code to extend its functionality.
\item Lack of visualization tools at each step of analysis makes it hard for the
  researchers to check the quality of result. This is a significant issue
  especially since there are many detailed parameters that may have a huge
  effect on the final output of the CNMF algorithm.
\item The lack of a batch-processing pipeline, as well as an easy way to
  visualize the results across multiple datasets, limits the applicability of
  the algorithm to the analysis of large scale experiments.
\end{inparaenum}
Thus, to address these difficulties, we will develop an analysis pipeline that
provides a user--friendly interface to the CNMF algorithm, and provides
visualization tools as well as batch-processing capability, integrated into a
single analysis package.

In summary, the proposed studies will investigate whether the temporal window of
memory linking is affected by the order of events, which is important for the
ethological implication of memory linking. Additionally, the proposed studies
will investigate how memory linking happens on the neural ensemble level. Taken
together, the results from these studies will help us understand more about the
memory linking phenomenon and episodic memories in general.

\end{document}
