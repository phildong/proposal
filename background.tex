\documentclass[master.tex]{subfiles}

\begin{document}

\section*{A. Significance}

Understanding how temporally distinct memories can be related together is
essential to understanding episodic memory, since the ability to associate
different episodes across long periods of times is essential to forming
memories. It is generally believed that hippocampus make an important
contribution to episodic memories in rodents. Traditionally, studies have
focused on hippocampal coding within episodes, which are usually testing
sessions of a behavior task repeated across days. Most notably, we have
extensive knowledge of how space is coded in familiar settings, as well as how
time is coded in well-learned time-dependent tasks. On the other hand, it is
found that different population of hippocampal neurons are engaged in memories
of different contexts across days. Thus the population of neurons encoding a
specific memory is termed the ensemble of said memory, and it is found that
artificially stimulating an ensemble is sufficient to drive the retrieval of the
corresponding memory and elicit behavior response. Interestingly, it is found
that ensembles of same familiar environments drift across time, in that
different population of neurons are recruited to encode the same environment at
different times. Moreover, the number of neurons that are shared between two
ensembles, or, the ``overlap'' between two ensembles, varies in a time-dependent
manner, so that the overlap of ensembles encoding memories of the same
environment is higher if the memories are separated with shorter time interval.
However, it was unclear whether overlaps of ensembles encoding different
environments have a similar time-dependent property, and whether such difference
in overlaps have any behavior correlates. If different ensembles of memories
merely reflect their difference in features, it would be surprising to find any
time modulation of the overlaps between memories encoding different
environments.

Recently, it has been found in rodent hippocampus that the neuronal ensembles of
two distinct contextual memories separated by 5 hours time interval has more
overlapping cells than those separated by 2 days or 7 days. Interestingly,
subsequent fear conditioning in the later context induce elevated freezing level
in the former context when the two contexts are separated by 5 hours, indicating
a transfer of fear memory from the second context to the first. Such results
suggest a linking of two temporally distinct memories through overlapping
neuronal ensembles. Following these findings, two models have been proposed to
explain the phenomenon of memory linking: On cellular and circuit level, It has
been hypothesized that memory linking happens through excitability mechanism,
where the ensemble neurons of first memory sustain an elevated excitability
during the memory linking time window, and thus are more likely to be recruited
during the encoding of the second memory, facilitating the linking of the two
memory; At the same time, from a conceptual and computational aspect, temporal
context model suggests that memory linking is a peculiar case of a more general
temporal context framework, which argues that features of memories are
associated with an ever-drifting temporal context, and all recollection of
episodic memories, as well as formation of semantic memories, happen through the
retrieval of the associated temporal context.

However, two important and inter-related questions remain unclear for the memory
linking phenomenon:
\begin{inparaenum}[a)] \bfseries
\item whether and how the temporal order of the experiences affect the temporal
  window of memory linking.
\item whether and how negative emotional value of the experiences affect the
  temporal window of memory linking.
\end{inparaenum}
Regarding the first question, the excitability hypothesis would predict no
effect of temporal order, \textit{i.e.} a memory would link to a past memory
equally well as to a future memory, whereas temporal context model would predict
a stronger prospective memory linking, \textit{i.e.} a memory would share more
overlapping ensemble and have stronger association with the other memory that
happens later in time. Regarding the second question, the excitability
hypothesis would predict an effect of negative emotional value on prospective
memory linking, so that a memory that has a more negative value, comparing to a
neutral memory, would have a stronger association with memories in the future,
but not in the past. Meanwhile, the temporal context model would fail to provide
a prediction regarding the second question since emotional valence has not been
integrated into the model.

However, from a ethological perspective, the predictions regarding the two
questions would be different from those predicted by the two existing models.
Specifically:
\begin{inparaenum}[a)]
\item relating a memory to past experiences is more beneficial than relating a
  memory to future experiences, since only past experiences may have a causal
  role which is important to learn.
\item relating a highly traumatic memory to other experiences (especially past
  experiences) is more beneficial than relating a neutral memory to others, so
  that the animal may learn to avoid the same situation in the future.
\end{inparaenum}
An important example of such perspective is conditioned taste aversion, where
the animals learned to associate the negative experience (sickness) with a past
experience (consumption of food), and such association depends on the valence of
the negative valence (how much sickness was induced).

Thus, the main goal of this proposal is to study the effect of temporal order
and emotional value on memory linking. The study would help us understand the
mechanism and behavioral significance of memory linking, and may extend our
knowledge on episodic memory in general.

In addition, the analysis of neural recording data in memory linking experiments
have been limited to comparing overlapping ensemble cells that are active during
recording sessions. Such analysis provided a simple estimation of similarities
between ensembles and successfully supported behavior data. However, it reduces
the time dimension of each recording session to a binary, ``active-or-not''
representation, precluding any analysis on the structure of the ensemble within
session. This limitation is mainly due to the task-free and one-trial-learning
nature of memory linking, where there is no task variables to align the
recording data to, nor is there enough time for place cells to be formed and
detected. Dimension reduction approaches, or more specifically principal
component analysis (PCA) is a very useful tool in such circumstances, since it
can transform higher dimension recording data to lower dimension, temporally
structured components in an unsupervised manner. Thus another goal of the
presented proposal is to apply PCA analysis to neural recording data during
memory linking experiments. Such analysis could uncover the underlying
structures of memory ensembles, and help us understand the nature of memory
linking on the ensemble level.


\end{document}
