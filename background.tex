\documentclass[master.tex]{subfiles}

\begin{document}

\section*{A. Scientific Premise}

It has been found in mice that two contextual memories happened close in time
can be linked together, in that the ensembles of the two memories share a larger
proportion of overlapping neurons, and fear conditioning in one of the contexts
result in freezing behavior in the other. However, the temporal window, as well
as the neuronal mechanism of such phenomenon remains under--studied.


\section*{B. Background and Significance}

Understanding how distinct memories can be related together is essential to
understanding episodic memory, since the ability to associate different episodes
across long periods of time is essential to forming memories. It is generally
believed that different memories are encoded with different population of
neurons, which are reactivated during memory retrieval. Such specific population
of neurons are termed the ensemble of said memory. Indeed, it has been found
that artificially stimulating an ensemble is sufficient to drive the retrieval
of the memory and elicit behavior response \cite{ramirez_creating_2013}. Thus,
it can be speculated that two distinct memories can be associated together by
sharing a proportion of neurons that is common to both ensembles, so that
activation of one ensemble might trigger the activation of the other as well,
potentially through pattern completion mechanism. Consistent with this idea, it
is found that two types of aversive memories can be associated together by
repeatedly presenting the two stimulus that triggers them at the same time, such
that further presentation of one stimulus will trigger the behavior response of
the other. Importantly, the overlap between two ensembles encoding the two
aversive memories is larger than those in a control group that did not undergo
the paired presentation of stimulus, and that artificially suppressing the
activities of the neuron population shared between two ensembles disrupted the
association between the two aversive memories, without affecting the independent
recall for each of them \cite{yokose_overlapping_2017}. Besides explicit pairing
of two memories by triggering the recall of them at the same time, it is also
found that two fear memories can be associated together if they are encoded
close in time. Specifically, two auditory fear conditioning memories can be
associated together if they are encoded with a 6 hours time interval in between,
but not when they are separated by 24 hours. Consistent with previous findings,
the ensembles of the two memories that were associated together had a higher
proportion of overlap between the two ensembles \cite{rashid_competition_2016}.
It is hypothesized that such time-dependent association of different memories is
mediated by an excitability-based memory allocation mechanism, where the neurons
encoding an earlier memory sustain an elevated level of excitability within
certain period of time, thus biasing the allocation of a later memory towards
the same population of neurons, resulting in an increase in overlap of the
ensembles of the two memories. Indeed, it has been found that artificially
manipulate the neurons encoding the first memory can bidirectionally bias the
allocation of the second memory and either rescue or disrupt the association
between the two memories \cite{rashid_competition_2016}. These studies have been
focusing on the amygdala in rodents, which is believed to play an important role
in the formation of associative fear memories. Taken together, these studies
suggest that aversive memories can be associated together in a time-dependent
manner. It was unknown, however, whether such association can occur naturally
with memories that have neutral emotional valence, which is key to the question
of how episodes of memories that does not have negative emotional valence can be
linked together.

Recently, it has been found in rodent hippocampus that the neuronal ensembles of
two neutral contextual memories separated by 5 hours has more overlapping cells
between ensembles than those separated by 2 days or 7 days. Moreover, subsequent
fear conditioning in the later context induce elevated freezing level in the
former context when the two contexts are separated by 5 hours, indicating a
transfer of fear memory from the second context to the first. Such results
suggest a linking of two temporally distinct memories through overlapping
neuronal ensembles, and the phenomenon is termed memory linking
\cite{cai_shared_2016}. In line with this finding, it has also been found that
the representation of a familiar environment in rodent hippocampus drift across
time, so that both the similarity between ensembles and the stability of spatial
map of the same environment decay as a function of temporal distance
\cite{mankin_neuronal_2012, ziv_long-term_2013-1}. Such drift reflect a
spontaneous turn--over of neural ensembles, which, when combined with the memory
allocation hypothesis, could serve as a potential explanation of the phenomenon
of memory linking: if two memories happen close in time so that when the second
memory happens, neurons encoding the first memory still have elevated
excitability, the two memories will have larger overlap in ensembles since the
allocation of the second memory is biased towards the same neurons encoding the
first memory, thus the two memories will be linked together; If, however, the
two memories are separated with large time interval, so that the excitability of
neurons encoding the first memory return to baseline before second memory
happens, the two memories will have chance level overlap in ensembles due to
spontaneous turn--over of ensembles over time, thus they will not be linked.
Taken together, these results suggest that neutral memories can be linked in a
time-dependent manner, possibly mediated by a spontaneous drift of ensembles and
memory allocation mechanism.

However, the temporal window of memory linking, which is the maximum time
interval within which two memories can be linked together, remains
under--studied. This question is important to memory linking since it is closely
related to the ethological implication of memory linking --- It can be
speculated that memory linking may play a role in learning the relationships
between memory episodes, and the temporal window of memory linking reflects how
much information is gathered to form relational memories about a specific
episodes. One of the most important relationship between memory episodes is
causal relationship, where linking a traumatic experience with an earlier memory
may help the animal learn the causal relationship between memories and avoid
future traumatic events. An important example of such perspective is conditioned
taste aversion, where the animals learned to associate the negative experience
(sickness) with a past experience (consumption of food)
\cite{garcia_conditioned_1955}. Thus, retrospective memory linking, where the
animal link a traumatic experience with a previous neutral contextual memory,
should be more important than prospective memory linking, where the animal link
a traumatic experience with a neutral memory that happens later in time.
Specifically, the temporal window of memory linking should be longer for
retrospective memory linking than for prospective memory linking, since it helps
the animal to gather more information about the potential cause of the traumatic
event. However, the allocation hypothesis would predict the opposite --- Since
it has been shown that memories with negative valence increase neuron
excitability \cite{rashid_competition_2016}, it is expected that neurons engaged
in a negative memory sustain an elevated level of excitability longer than those
in a neutral memory. Thus, for a negative memory, the allocation hypothesis
would predict a longer temporal window for prospective memory linking comparing
to retrospective memory linking. In order to study the temporal window of
prospective and retrospective memory linking, we have carried out preliminary
behavior studies. The results suggest a longer temporal window for retrospective
memory linking than for prospective memory linking. Thus, we hypothesize that
retrospective memory linking has a longer temporal window than prospective
memory linking.

Moreover, it is unclear how does memory linking happen. As mentioned before, we
have ethological speculations and preliminary data that cannot be fully
addressed by the memory allocation hypothesis. We will focus on studying the
mechanism of memory linking from a neural coding perspective. Specifically, we
will focus on studying the neural correlate and ensemble dynamic of memory
linking. Firstly, we have preliminary data suggesting that the increase in
overlap of ensembles emerges after the encoding of the two memories, which is
contrary to what allocation hypothesis would predict. To confirm this
observation, we will compare the ensembles for each memory during encoding and
retrieval to see whether there is a change after encoding. We will also compare
the ensembles of one memory during encoding to the other during retrieval to see
whether memory ensembles is becoming more similar to each other. Secondly, one
the most important neural correlate of memory linking is the ensemble overlap
between the linked memories, since it has been shown that manipulating this
overlapping population of neurons directly and specifically affect memory
linking \cite{yokose_overlapping_2017}. However, it is unclear how each memory
contribute to the overlap. Thus, we will test whether the neurons engaged in the
encoding sessions of each memory contribute equally to the overlap of ensembles
during retrieval. Lastly, the allocation hypothesis predict that the activity
level of a neuron during the encoding of the first memory should determine the
likelihood of the said neuron being allocated to the second memory. However,
this correlation has never been explicitly tested in previous memory linking
studies. Thus, we will test whether the mean activity level during the encoding
of the first memory differs for the neurons that are allocated to the second
memories from those that are not. Taken together, these studies will give us
more insights in how memory liking happens on a neural ensemble level.

Lastly, calcium imaging with miniature microscope in behaving animals is an
important tool to study the neuronal dynamics in memory linking studies, due to
its capability to track large population of neurons across long period of time.
However, one of the challenges facing this technique is the analysis of imaging
data. Specifically, extracting calcium traces of individual neurons from raw
video is a difficult problem. Several approaches have been developed to address
this difficulty. Most notably, a variant of constrained non--negative matrix
factorization (CNMF) algorithm has been shown to work well on calcium imaging
data with miniature microscope \cite{pnevmatikakis_simultaneous_2016,
  zhou_efficient_2016}. However, in practice, the accessibility of such
approaches has been limited to the general scientific community that utilize
miniature microscope due to several reasons:
\begin{inparaenum}[a)]
\item The effect of different parameters in the mathematical model and the
  complications introduced by performance optimization of the codes can be hard
  to comprehend for researchers without mathematical or programming background.
\item A lack of pre--processing pipeline that specifically address various
  artifacts usually observed in miniature microscope requires the user to either
  rely on other software to conduct pre--processing steps or modify the existing
  codes to extend its functionality.
\item A lack of visualization tool at each step of analysis make it hard for the
  researchers to check the result and quality of computations. This is a
  significant issue especially since there are many detailed parameters that may
  have a huge effect on the final output of the algorithm.
\item The lack of batch-processing pipeline, as well as an easy way to visualize
  the results across multiple data--sets, limits the applicability of the
  algorithm to the analysis of large scale experiments.
\end{inparaenum}
Thus, to address these difficulties, the final goal of the proposal is to
develop an analysis pipeline that provide an user--friendly interface to the
CNMF algorithm, and provide visualization tools as well as batch-processing
capability all integrated in a single analysis package.

\end{document}
