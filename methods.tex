\documentclass[master.tex]{subfiles}

\begin{document}

\section*{C. Approach}

\begin{description}
\item[C.1. Instructions.]{}
\end{description}

Describe the overall strategy, methodology, and analyses to be used to accomplish the specific aims of the project. Unless addressed separately in Item 15 (Resource Sharing Plan), include how the data will be collected, analyzed, and interpreted as well as any resource sharing plans as appropriate.

Discuss potential problems, alternative strategies, and benchmarks for success anticipated to achieve the aims.

If the project is in the early stages of development, describe any strategy to establish feasibility, and address the management of any high risk aspects of the proposed work.

Point out any procedures, situations, or materials that may be hazardous to personnel and precautions to be exercised. A full discussion on the use of Select Agents should appear in Item 11, below.

As applicable, also include the following information as part of the Research Strategy, keeping within the three sections listed above: Significance, Innovation, and Approach.

\begin{description}
\item[C.2. Preliminary Studies for New Applications]{}
\end{description}

Preliminary Studies for New Applications: For new applications, include information on Preliminary Studies. Discuss the PD/PI's preliminary studies, data, and or experience pertinent to this application. Except for Exploratory/Developmental Grants (R21/R33), Small Research Grants (R03), and Academic Research Enhancement Award (AREA) Grants (R15), preliminary data can be an essential part of a research grant application and help to establish the likelihood of success of the proposed project. Early Stage Investigators should include preliminary data (however, for R01 applications, reviewers will be instructed to place less emphasis on the preliminary data in application from Early Stage Investigators than on the preliminary data in applications from more established investigators).

\end{document}